\documentclass[]{style/ceurart}
\sloppy

\usepackage{listings}
\lstset{breaklines=true}

\begin{document}

\copyrightyear{2024}
\copyrightclause{Copyright for this paper by its authors.
  Use permitted under Creative Commons License Attribution 4.0
  International (CC BY 4.0).}

\conference{CLEF 2024: Conference and Labs of the Evaluation Forum, September 9-12, 2023, Grenoble, France}

\title{DS@GT-CLEF: DinoV2 Transfer Learning for Snake Identification}

\author[1]{Anthony Miyaguchi}[
orcid=0000-0002-9165-8718,
email=acmiyaguchi@gatech.edu,
]
\cormark[1]
\author[1]{Murilo Gustineli}[
email=murilogustineli@gatech.edu,
]
\cormark[1]
\author[1]{Austin Fischer}[
email=afischer39@gatech.edu,
]
\author[1]{Ryan Lundqvist}[
email=rlundqvist@gatech.edu,
]

\address[1]{Georgia Institute of Technology, North Ave NW, Atlanta, GA 30332}
\cortext[1]{Corresponding author.}


\begin{abstract}
    This is the abstract of the paper. 
    It should be a brief summary of the contents of the paper.
\end{abstract}

\begin{keywords}
  LaTeX class \sep
  paper template \sep
  paper formatting \sep
  CEUR-WS
\end{keywords}


\maketitle

\section{Introduction}

Introduction to the problem and the approach.

\section{Related Work}

Overview of related work.

\section{Methodology}

A description of the data and methods.

For example, we use attention mechanisms \cite{vaswani2017attention} to improve the performance of a neural network on a classification task with multimodal data.

\section{Results}

This section includes the results of the experiments, primarily in tabular or graphical form.

\section{Discussion}

Discussion of the results and their implications.

\section{Future Work}

What would you do next?

\section{Conclusions}

Summary of the work and its contributions.

\section*{Acknowledgements}

Thank you to the DS@GT CLEF team for their support.

\bibliography{main}

% \appendix
% \section{Online Resources}

\end{document}